



{\setbeamercolor{background canvas}{bg=toccolor}
\begin{frame}

\Large How do RTEs relate to Deterministic Finite Automata (DFAs)?

\centering

\scalebox{0.6}{\input{fig-3}}

\end{frame}
}

\begin{frame}[t]{Example: How does a pattern predicate work?}
  \only<1-17>{sequence=}%
  \only<1-2>{\tt \textcolor{blue}{[13 2.0 6.0 4 "a" "an" "the" -5 2.0 3.0 4.0 7 8.0]}}%
  \only<3>{\tt [13 2.0 6.0 4 "a" "an" "the" -5 2.0 3.0 4.0 7 8.0]}%
  \only<4>{\tt [\colorbox{orange!30}{\Large 13} 2.0 6.0 4 "a" "an" "the" -5 2.0 3.0 4.0 7 8.0]}%
  \only<5>{\tt [13 \colorbox{orange!30}{\Large 2.0} 6.0 4 "a" "an" "the" -5 2.0 3.0 4.0 7 8.0]}%
  \only<6>{\tt [13 2.0 \colorbox{orange!30}{\Large 6.0} 4 "a" "an" "the" -5 2.0 3.0 4.0 7 8.0]}%
  \only<7>{\tt [13 2.0 6.0 \colorbox{orange!30}{\Large 4} "a" "an" "the" -5 2.0 3.0 4.0 7 8.0]}%
  \only<8>{\tt [13 2.0 6.0 4 \colorbox{orange!30}{\Large "a"} "an" "the" -5 2.0 3.0 4.0 7 8.0]}%
  \only<9>{\tt [13 2.0 6.0 4 "a" \colorbox{orange!30}{\Large "an"} "the" -5 2.0 3.0 4.0 7 8.0]}%
  \only<10>{\tt [13 2.0 6.0 4 "a" "an" \colorbox{orange!30}{\Large "the"} -5 2.0 3.0 4.0 7 8.0]}%
  \only<11>{\tt [13 2.0 6.0 4 "a" "an" "the" \colorbox{orange!30}{\Large -5} 2.0 3.0 4.0 7 8.0]}%
  \only<12>{\tt [13 2.0 6.0 4 "a" "an" "the" -5 \colorbox{orange!30}{\Large 2.0} 3.0 4.0 7 8.0]}%
  \only<13>{\tt [13 2.0 6.0 4 "a" "an" "the" -5 2.0 \colorbox{orange!30}{\Large 3.0} 4.0 7 8.0]}%
  \only<14>{\tt [13 2.0 6.0 4 "a" "an" "the" -5 2.0 3.0 \colorbox{orange!30}{\Large 4.0} 7 8.0]}%
  \only<15>{\tt [13 2.0 6.0 4 "a" "an" "the" -5 2.0 3.0 4.0 \colorbox{orange!30}{\Large 7} 8.0]}%
  \only<16>{\tt [13 2.0 6.0 4 "a" "an" "the" -5 2.0 3.0 4.0 7 \colorbox{orange!30}{\Large 8.0}]}%
  \only<17>{\tt [13 2.0 6.0 4 "a" "an" "the" -5 2.0 3.0 4.0 7 8.0]}%
 \only<18>{\textcolor{blue}{Decision procedure is $O(n)$,\Emph{independent of syntactical complexity} of the RTE.}}%

 \begin{columns}
   \begin{column}{0.4\textwidth}
     \only<2>{

       \bigskip

       Does the sequence match the pattern?
       \textcolor{greeny}{${(Long \cdot ( Double^{+} ~\vee String^{+}))^{+}}$}

       \bigskip

     
       \textcolor{blue}{  (DFA) \Emph{Deterministic Finite Automaton}}.
       
       \bigskip
       
       Construct a DFA from an RTE? \Emph{Compile-time}}%
     \only<3,17,18>{\textcolor{greeny}{\LARGE $pattern={{(Long \cdot ( Double^{+} ~\vee String^{+}))^{+}}}$}\leavevmode\\[2cm]}%
     \only<4,7,11,15>{\textcolor{greeny}{\LARGE $pattern={{(\Emph{Long} \cdot ( Double^{+} ~\vee String^{+}))^{+}}}$}\leavevmode\\[2cm]}%
     \only<5,6,12-14,16>{\textcolor{greeny}{\LARGE $pattern={{(Long \cdot ( \Emph{Double}^{+} ~\vee String^{+}))^{+}}}$}\leavevmode\\[2cm]}%
     \only<8,9,10>{\textcolor{greeny}{\LARGE $pattern={{(Long \cdot ( Double^{+} ~\vee \Emph{String}^{+}))^{+}}}$}\leavevmode\\[2cm]}%
     \only<3-16>{\textcolor{orange}{\LARGE Does the sequence match the pattern?} \Emph{Run-time}}%
     \only<17>{\Emph{\LARGE Yes, it's a match!}}%
     \only<18>{\includegraphics[height=3.cm]{exploding-head}}
   \end{column}
   \begin{column}{0.6\textwidth}
     \only<2,18>{\scalebox{0.95}{\input{fig-3}}}%
     \only<3>{\scalebox{0.95}{\input{fig-3-0}}}%
     \only<4>{\scalebox{0.95}{\input{fig-3-0-int}}}%
     \only<5>{\scalebox{0.95}{\input{fig-3-1-float}}}%
     \only<6>{\scalebox{0.95}{\input{fig-3-2-float}}}%
     \only<7>{\scalebox{0.95}{\input{fig-3-2-int}}}%
     \only<8>{\scalebox{0.95}{\input{fig-3-1-String}}}%
     \only<9>{\scalebox{0.95}{\input{fig-3-3-String}}}%
     \only<10>{\scalebox{0.95}{\input{fig-3-3-String}}}%
     \only<11>{\scalebox{0.95}{\input{fig-3-3-int}}}%
     \only<12>{\scalebox{0.95}{\input{fig-3-1-float}}}%
     \only<13>{\scalebox{0.95}{\input{fig-3-2-float}}}%
     \only<14>{\scalebox{0.95}{\input{fig-3-2-float}}}%
     \only<15>{\scalebox{0.95}{\input{fig-3-2-int}}}%
     \only<16>{\scalebox{0.95}{\input{fig-3-1-float}}}%
     \only<17>{\scalebox{0.95}{\input{fig-3-2}}}%
   \end{column}
 \end{columns}
\end{frame}

\begin{frame}{Deterministic (DFA) vs Non-deterministic (NFA)}

  Suppose sequence = \code{[2,  \colorbox{orange!30}{\Large 3}, 5.6F]}

  \medskip

  \only<1>{\scalebox{0.95}{\input{fig-nfa-subtype}}}%
  \only<2>{\scalebox{0.95}{\input{fig-dfa-subtype}}}%
\end{frame}

\newsavebox\missiledemo
\begin{lrbox}{\missiledemo}
  \begin{minipage}{15cm}
    \begin{lstlisting}[style=conjClojure,numbers=none]
(defn missile-demo [input-seq]
  (destructuring-case input-seq
    [[a b]        {a Boolean b (or String Boolean)}]
    (rename-file a b)

    [[a b]        {a Boolean b (or String (satisfies int?))}]
    (delete-file a b)

    [[a b c & d]  {[a d] Boolean b String c (satisfies int?)}]
    (copy-file a b c d)

    [[a b]        {a Boolean b (or String Long)}]
    (~~launch-the-missiles~~ a b)))
\end{lstlisting}

  \end{minipage}
\end{lrbox}

\newsavebox\missiledemoone
\begin{lrbox}{\missiledemoone}
  \begin{minipage}{15cm}
    \input{missiledemoone}
  \end{minipage}
\end{lrbox}
\newsavebox\missiledemotwo
\begin{lrbox}{\missiledemotwo}
  \begin{minipage}{15cm}
    \begin{lstlisting}[style=conjClojure,numbers=none]
(defn missile-demo [input-seq]
  (destructuring-case input-seq
    [[a b]]
    (rename-file a b)

    [[a b]]
    (delete-file a b)

    [[a b c & d]]
    (copy-file a b c d)))
\end{lstlisting}

  \end{minipage}
\end{lrbox}



\begin{frame}{Example, \code{destructuring-case} macro}

\only<1,2>{Macro \code{destructuring-case} evaluates clause matching the correct template.}
\only<2>{\\And discriminates \emph{based on types} of imbedded values.}
  
\medskip

\only<1>{\usebox\missiledemotwo}%
\only<2>{\usebox\missiledemoone}%
\only<3>{\usebox\missiledemo}%

\only<3>{We would like to know \Emph{at compile time} whether there is a possible value of \code{input-seq}
  which will \Emph{launch the missiles}?}
\end{frame}

\begin{frame}{Code converted to DFA}
  \begin{columns}
  
\begin{column}{.6\textwidth}
\begin{minipage}[t]{\linewidth}
  \includegraphics[width=\textwidth, keepaspectratio]{missile-demo-dfa}
\end{minipage}
\end{column}

    \begin{column}{.4\textwidth}
      \begin{align*}
        t_0 &= Boolean\\
        t_1 &= Byte \cup Integer \cup Long \cup Short\\
        t_2 &= String\\
        t_3 &= Long\\
        t_4 &= Byte \cup Integer \cup Short
      \end{align*}
    \end{column}

  \end{columns}

  Macro-expansion warning: \\
  \textcolor{red}{\texttt{Unreachable code: (launch-the-missiles a b)}}\\
  because DFA has no path to \code{launch-the-missiles}!
\end{frame}

 % matching (a 1 1.0 b "a" "an" "the" -5 2 22 222 d 2/3)

\begin{frame}{DFA Library \code{xymbolyco}}
  Symbolic Finite Automata Library
  \begin{itemize}
    \item RTE to DFA
    \item DFA to RTE
    \item Pattern matching
    \item Minimization
    \item Union, Intersection, Emptiness check
    \item Need to test \code{xymbolyco}
    \begin{itemize}
    \item Random generation of DFA
    \end{itemize}

  \end{itemize}
\end{frame}

\section{Generating random DFA}
{\setbeamercolor{background canvas}{bg=toccolor}
\begin{frame}{Generating random DFA}

  \begin{itemize}
  \item We can construct DFA from RTE
  \item ... so generate random RTE (as seen earlier)
  \item ... and construct DFA
  \end{itemize}
  
\end{frame}
}

\begin{frame}{Successful: Generation of DFA from RTE}
  \begin{columns}
    \begin{column}{0.5\textwidth}
      \includegraphics[width=\textwidth, height=0.9\textheight, keepaspectratio]{balanced-rte}
    \end{column}
    \begin{column}{0.5\textwidth}
      \includegraphics[width=\textwidth, height=0.9\textheight, keepaspectratio]{balanced-dfa}
    \end{column}
  \end{columns}
\end{frame}

\begin{frame}{Dubious: Generation of DFA from RTE}
  \begin{columns}
    \begin{column}{0.4\textwidth}
      \includegraphics[width=\textwidth, height=0.9\textheight, keepaspectratio]{rte-naive-edge}
    \end{column}
    \begin{column}{0.6\textwidth}
      \begin{itemize}
      \item<1->{Trivial DFA. \includegraphics[width=3cm, keepaspectratio]{dfa-naive-edge}}
      \item<2->{Non-uniformity caused by so-called annihilators}
      \item<3->{Whenever an input $X=\emptyset$ then $X \cap Y = \emptyset$.}
      \item<4->{Thus entire tree generating $Y$ is \textcolor{red}{in vain.}}
      \end{itemize}
    \end{column}
  \end{columns}
\end{frame}





\begin{frame}{Proposed fix: Balanced Tree Generation}
  \begin{columns}[t]
    \begin{column}{0.4\textwidth}
      \textcolor{red}{UNIFORM\\ \emph{Portrait} --- narrow and deep}

      \centering
      
      \begin{minipage}{0.75\textwidth}
        \includegraphics[width=\textwidth, height=0.8\textheight, keepaspectratio]{rte-narrow-and-deep}
      \end{minipage}%
      
    \end{column}
    \begin{column}{0.6\textwidth}
      \textcolor{greeny}{BALANCED\\ \emph{Landscape} --- shallow and wide}

      \centering
      
      \begin{minipage}{0.85\textwidth}
        \includegraphics[width=\textwidth, height=0.6\textheight, keepaspectratio]{rte-shallow-and-wide}
      \end{minipage}%
    \end{column}
  \end{columns}
\end{frame}

\begin{frame}{Requirements}
  \begin{itemize}

  \item If we want \emph{large} DFAs, we necessarily \textcolor{gold}{should} generate \emph{large} RTE (AST).

  \item However, this is not sufficient.  Many large RTEs generate small/trivial DFAs.

  \item We must generate \textcolor{greeny}{Portrait} not \textcolor{red}{Landscape} aspect ratios.
  \end{itemize}
\end{frame}
