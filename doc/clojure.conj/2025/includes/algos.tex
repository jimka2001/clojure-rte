
{\setbeamercolor{background canvas}{bg=toccolor}
\begin{frame}

\Large {Algorithms for Balanced Generation}
  
\end{frame}
}

\begin{frame}{Tree-Split Algorithm}{Divide and Conquer}

  To generate tree having $n$ leaves, find $p=pivot$ between 1 and
  $n$; build tree with $p$ leaves on left and $n-p$ leaves on right.
  Repeat recursively.
  \begin{columns}
    \begin{column}{.5\textwidth}
      \only<1>{\includegraphics[height=.8\textheight]{tree-split-1}}%
      \only<2>{\includegraphics[height=.8\textheight]{tree-split-2}}%
      \only<3>{\includegraphics[height=.8\textheight]{tree-split-3}}%
      \only<4>{\includegraphics[height=.8\textheight]{tree-split-4}}%
      \only<5>{\includegraphics[height=.8\textheight]{tree-split-5}}%
      \only<6->{\includegraphics[height=.8\textheight]{tree-split-6}}%
    \end{column}\hspace{-1cm}    
    \begin{column}{.5\textwidth}
      \only<7>{Question is how to produce the pivot?
        \begin{itemize}
        \item Gaussian
        \item Inverse Gaussian
        \item Linear
        \item Lopsided Comb
        \end{itemize}
      }
    \end{column}
  \end{columns}
\end{frame}



\newsavebox\clojtreesplitcore
\begin{lrbox}{\clojtreesplitcore}
  \begin{minipage}{15cm}
    \begin{lstlisting}[style=conjClojure]
(defn tree-split-rte [leaves pivot]
  (if (= 1 leaves)
    (rand-nth inhabited-leaves)
    (let [rte (let [left-size (pivot leaves)]
                (letfn [(left  [] (tree-split-rte left-size pivot))
                        (right [] (tree-split-rte (- leaves left-size) pivot))
                        (mid   [] (tree-split-rte leaves pivot))]
                  (weighted-case 20 (list :cat (left) (right))
                                 30 (list :and (left) (right))
                                 30 (list :or (left) (right))
                                 5  (list :* (mid))
                                 15 (list :not (mid)))))
          leaf-count (rte/count-leaves rte)]
      rte)))
\end{lstlisting}

  \end{minipage}
\end{lrbox}
\begin{frame}{RTE Tree-Split Generator}
  \usebox\clojtreesplitcore
\end{frame}

\newsavebox\clojtreesplit
\begin{lrbox}{\clojtreesplit}
  \begin{minipage}{15cm}
    \begin{lstlisting}[style=conjClojure]
;; lopsided comb
(defn comb-rte [leaves]
  (tree-split-rte leaves
                  (fn [n]
                     (case n
                       (1 2 3)  1
                       2))))

;; linear split
(defn tree-split-rte-linear [leaves]
  (tree-split-rte leaves 
                  (fn [n] (max 1 (rand-int n)))))
\end{lstlisting}

  \end{minipage}
\end{lrbox}
\newsavebox\clojtreesplittwo
\begin{lrbox}{\clojtreesplittwo}
  \begin{minipage}{15cm}
    \begin{lstlisting}[style=conjClojure,numbers=none]
;; Gaussian split
(defn tree-split-rte-gaussian [leaves]
  (tree-split-rte leaves 
                  (fn [n]
                    (case n
                      (1 2 3) 1
                      (max 1 (+ (rand-int (quot n 2))
                                (rand-int (quot n 2))))))))

;; inverse Gaussian split
(defn tree-split-rte-inv-gaussian [leaves]
  (tree-split-rte leaves
                  (fn [n]
                    (let [r (max 1 (rand-int (quot n 2)))]
                       (if (= 0 (rand-int 2))
                         r
                         (- n r))))))
\end{lstlisting}

  \end{minipage}
\end{lrbox}



\begin{frame}{RTE Tree-Split 4-variants}
  \begin{columns}
    \begin{column}{0.65\textwidth}
  \only<1>{\usebox\clojtreesplit}%
  \only<2>{\usebox\clojtreesplittwo}
    \end{column}\hspace{-1cm}
    \begin{column}{0.35\textwidth}
      \only<2>{\includegraphics[width=\textwidth]{bell_vs_shifted_inverted_bell}}
    \end{column}
  \end{columns}
\end{frame}

\begin{frame}{Flajolet Algorithm}{Generate a Skeleton}

  Inserting 0 \ldots 9 in random order into Binary Search Tree: \code{[5 4 2 8 7 9 0 3 6 1]}
  \begin{columns}
    \begin{column}{.5\textwidth}
  \only<1>{\includegraphics[height=.8\textheight]{flajolet-1-5}}%
  \only<2>{\includegraphics[height=.8\textheight]{flajolet-2-45}}%
  \only<3>{\includegraphics[height=.8\textheight]{flajolet-3-245}}%
  \only<4>{\includegraphics[height=.8\textheight]{flajolet-4-2458}}%
  \only<5>{\includegraphics[height=.8\textheight]{flajolet-5-24578}}%
  \only<6>{\includegraphics[height=.8\textheight]{flajolet-6-245789}}%
  \only<7>{\includegraphics[height=.8\textheight]{flajolet-7-0245689}}%
  \only<8>{\includegraphics[height=.8\textheight]{flajolet-8-02345789}}%
  \only<9>{\includegraphics[height=.8\textheight]{flajolet-9-023456789}}%
  \only<10->{\includegraphics[height=.8\textheight]{flajolet-10-0123456789}}%
    \end{column}\hspace{-1cm}    
    \begin{column}{.5\textwidth}
      \only<11>{After generating skeleton, we fit an RTE to that skeleton.    }
    \end{column}
  \end{columns}
\end{frame}



\newsavebox\clojflajolet
\begin{lrbox}{\clojflajolet}
  \begin{minipage}{15cm}
    \begin{lstlisting}[style=conjClojure,numbers=none]
(defn flajolet-rte-by-size [leaves]
  (letfn [(insert [tree d]
            (if (empty? tree)
              [d nil nil]
              (let [[n left right] tree]
                (if (< d n)
                  [n (insert left d) right]
                  [n left (insert right d)]))))
          ...]
    
    (let [skeleton (reduce insert
                           nil
                           (shuffle (range (dec leaves))))]
      
      ...)))
\end{lstlisting}

  \end{minipage}
\end{lrbox}
\newsavebox\clojflajolettwo
\begin{lrbox}{\clojflajolettwo}
  \begin{minipage}{15cm}
    \begin{lstlisting}[style=conjClojure,numbers=none]
(defn flajolet-rte-by-size [leaves]
  (letfn [(insert [tree d]
            ...)
          (to-rte [tree]
            (cond (> (random) 0.90) ;; 10% probability
                  (weighted-case 50 (list :not (to-rte tree))
                                 50 (list :* (to-rte tree)))
                  
                  (empty? tree)
                  (rand-nth inhabited-leaves)

                  :else
                  (let [[_ left right] tree]
                    (weighted-case
                       34 (list :cat (to-rte left) (to-rte right))
                       33 (list :or  (to-rte left) (to-rte right))
                       33 (list :and (to-rte left) (to-rte right))))))]
    (let [skeleton ...]
      (to-rte skeleton))))
\end{lstlisting}

  \end{minipage}
\end{lrbox}
\begin{frame}{RTE Flajolet Generator}
  \only<1>{\usebox\clojflajolet}
  \only<2>{\usebox\clojflajolettwo}
\end{frame}
