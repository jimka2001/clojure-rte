

\subsection{Determinism: Type Partitioning}

{  %% chapter slide
%  \setbeamercolor{background canvas}{bg=sectioncolor}
\begin{frame}{\Challenge{4} A Sufficient Type Partitioning}{MDTD: Maximal Disjoint Type Decomposition}
  \only<1>{Given overlapping (super)types $A_1\ldots A_n$ in an RTE}%
  \only<2>{Construct (sub) types $\typevar_1\ldots \typevar_m$}%
  \only<3>{Which are disjoint \Emph{by construction}, even if subtype relation is unknown.}%
  \only<4>{Assures that $\deriv{\typevar}{r}$ is computable.}%
  \only<5>{But transitions may be \Emph{indeterminate} and \Emph{unsatisfiable}.}%
  \begin{columns}
    \begin{column}{0.5\textwidth}
      \only<1,2>{\scalebox{0.98}{\input{pgf-a1-a8.tex}}}%
      \only<3->{\includegraphics[scale=0.18]{venn-x1-x13.png}}%
    \end{column}%
    \begin{column}{0.5\textwidth}
      \only<2->{\scalebox{0.98}{\input{pgf-x1-x13.tex}}}%
    \end{column}%
  \end{columns}%
\end{frame}


%% \begin{frame}{Are there dire Consequences?}

%%   \begin{itemize}
%%   \item   $\{\typevar_1, \typevar_2, \ldots, \typevar_{10}\}$ mutually disjoint by construction.
%%   \item   However, some $\typevar_i$ may be unknowingly vacuous.
%%   \item   Nevertheles $\emptyset$ is disjoint with every set, including itself.
%%   \item   Thus transitions may be \Emph{indeterminate/unsatisfiable}.
%%   \item   What is the consequence transitions which are  \Emph{indeterminate} and \Emph{unsatisfiable}?
%%   \end{itemize}
%% \end{frame}

\begin{frame}{DFA with indeterminate transitions: $(int \cdot str^{*} \cdot even)^{*}$}
  \begin{columns}
    \begin{column}{0.5\textwidth}
      \begin{align*}
        p_0 &= (int \cdot str^{*} \cdot even)^{*}\\
        p_1 &= \emptyset\\
        p_2 &= \tystring^{*} \cdot \tyeven \cdot (\tyint \cdot \tystring^{*} \cdot \tyeven)^{*}\\
        p_3 &= \tystring^{*}\!\!\cdot\! \tyeven\!\cdot\! (\tyint\! \cdot\! \tystring^{*}\!\!\cdot\! \tyeven)^{*}\! \reor (\tyint\! \cdot\! \tystring^{*}\!\!\cdot\!\! \tyeven)^{*}
      \end{align*}



      \includegraphics[width=0.9\textwidth,trim={1.4cm 1.2cm 1.4cm 0.8cm},clip=true]{reclojure-2025-sink-example-2}
    \end{column}
    \begin{column}{0.5\textwidth}
      \begin{align*}
        t_1 &= \Sigma\\
        t_2 &= int  \\
        t_3 &= \compl{int}\\
        t_4 &= \textcolor{red}{str \cap even}\\
        t_5 &= \compl{str} \cap even\\
        t_6 &= str \cap \compl{even}\\
        t_7 &= (int \cap \compl{even}) \cup (str \cap \compl{even})\\
        t_8 &= \textcolor{red}{\compl{int} \cap \compl{str} \cap even}\\
        t_9 &=(int \cap even) \cup (str \cap even)\\
        t_{10} &=\compl{str} \cap \compl{even}\\
        t_{11} &=\compl{int} \cap \compl{str} \cap \compl{even}
      \end{align*}
    \end{column}
  \end{columns}
\end{frame}


