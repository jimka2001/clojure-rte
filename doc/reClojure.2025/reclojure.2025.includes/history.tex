\section{Background}
{  %% chapter slide
  \setbeamercolor{background canvas}{bg=chaptercolor}
\begin{frame}{Background}
  \centering
  %% image from https://patimes.org/history-and-its-relevance-to-public-administration/  
  \includegraphics[width=0.8\textwidth]{reclojure-history.jpg}
\end{frame}
}

\begin{frame}{Goal}
  \begin{itemize}
  \item To efficiently recognize \emph{regular patterns} in mixed-type sequences.
  \item Primarily a research topic
    \begin{itemize}
    \item \ldots but extract useful libraries if possible.
    \end{itemize}
  \item To support in various programming languages:
    \begin{itemize}
    \item Scala as \code{Seq[Any]}
    \item Python as \code{list}, \code{generator}
    \item Clojure as \code{seq}
    \item Common Lisp as \code{SEQUENCE}
    \end{itemize}

  \end{itemize}
\end{frame}



\begin{frame}{Publication History}
  \begin{itemize}
  \item Common Lisp
    
    \begin{itemize}
    \item \textit{Type-Checking Heterogeneous CL Sequences} [Newton, Demaille, Verna] 2016 ELS
    \item \textit{Recognizing hetergeneous sequences by rational type
      expression}, [Newton, Verna] 2018 Workshop on Meta-Programming
      Techniques and Reflection.
    \item PhD Thesis 2018 for theoretical, implementation, and performance details
      \url{https://www.lrde.epita.fr/wiki/Publications/newton.18.phd}
      
    \end{itemize}
  \item Generalized beyond Common Lisp
    \begin{itemize}
    \item \textit{A Portable, Simple, Embeddable Type System} [Newton, Pommellet] 2021 ELS
      
    \item \textit{An Elegant and Fast Algorithm for Partitioning Types} [Newton] 2023 ELS
    \item \textit{Type-Checking Heterogeneous Sequences in a Simple Embeddable Type System} [Newton] 2025 PADL
    \end{itemize}
  \item   RTE Available at: \url{github.com/jimka2001/python-rte}.
  \end{itemize}
\end{frame}



\begin{frame}{Motivating Demo}

\end{frame}
