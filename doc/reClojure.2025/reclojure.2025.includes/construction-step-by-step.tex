
\begin{frame}{RTE Construction of \nodecirc{$p_0$}}
  \begin{columns}
    \begin{column}{0.45\textwidth}
      Step 1: construct transitions from~\nodecirc{$p_0$} using Brzozowski derivatives of RTE~$p_0$.

      \[p_0=(int \cdot str^{*} \cdot even)^{*}\,.\]


      \only<1-2>{\includegraphics[width=0.9\textwidth,trim={1.4cm 1.2cm 1.4cm 0.8cm},clip=true]{reclojure-2025-sink-example-pre}}%
      \only<3>{\includegraphics[width=0.9\textwidth,trim={1.4cm 1.2cm 1.4cm 0.8cm},clip=true]{reclojure-2025-sink-example-pre-p1}}%
      \only<4->{\includegraphics[width=0.9\textwidth,trim={1.4cm 1.2cm 1.4cm 0.8cm},clip=true]{reclojure-2025-sink-example-p0}}
    \end{column}
    \begin{column}{0.55\textwidth}
      \begin{itemize}
      \item<1->{What are the \emph{first} types of $p_0$: $\{\tyint\}$.}
      \item<2->{Partition $\Sigma$ by $\{\tyint\} \to \{\tyint,\tynot{\tyint}\}$.}
      \item<3->{$\deriv{\tynot{\tyint}}{p_0} = \underbrace{\emptyset}_{p_1}$}
      \item<4->{$\deriv{\tyint}{p_0} = \underbrace{\tystring^{*} \cdot \tyeven \cdot (\tyint \cdot \tystring^{*} \cdot \tyeven)^{*}}_{p_2}$}
      \end{itemize}
    \end{column}
  \end{columns}
\end{frame}

\begin{frame}{RTE Construction  of \nodecirc{$p_0$}}
  \begin{columns}
    \begin{column}{0.5\textwidth}
      \includegraphics[width=0.9\textwidth,trim={1.4cm 1.2cm 1.4cm 0.8cm},clip=true]{reclojure-2025-sink-example-p0}

      \begin{align*}
        p_0 &= (int \cdot str^{*} \cdot even)^{*}\\
        p_1 &= \deriv{\tynot{\tyint}}{p_0} = \emptyset\\
        p_2 &= \deriv{\tyint}{p_0}    
        = \tystring^{*} \cdot \tyeven \cdot (\tyint \cdot \tystring^{*} \cdot \tyeven)^{*}
      \end{align*}

    \end{column}
    \begin{column}{0.5\textwidth}
      \begin{align*}
        t_2 &= \tyint  \\
        t_3 &= \compl{\tyint}
      \end{align*}
    \end{column}
  \end{columns}
\end{frame}
\begin{frame}{RTE Construction of \nodecirc{$p_1$}}
  \begin{columns}
    \begin{column}{0.45\textwidth}
      Step 2: construct transitions from~\nodecirc{$p_1$} using Brzozowski derivatives of RTE~$p_1$.

      \[p_1=\emptyset\,.\]

      \includegraphics[width=0.9\textwidth,trim={1.4cm 1.2cm 1.4cm 0.8cm},clip=true]{reclojure-2025-sink-example-p1}
    \end{column}
    \begin{column}{0.55\textwidth}
      \begin{itemize}
      \item<1->{What are the \emph{first} types of $p_1$: $\{\emptyset\}$.}
      \item<2->{Partition $\Sigma$ by $\{\emptyset\} \to \{\Sigma\}$.}
      \item<3->{$\deriv{\Sigma}{\emptyset} = \emptyset$}
      \item<4->{      \begin{align*}
        t_1 &= \Sigma\\
        t_2 &= \tyint  \\
        t_3 &= \compl{\tyint}
      \end{align*}}
      \end{itemize}
    \end{column}
  \end{columns}
\end{frame}



\begin{frame}{RTE Construction of \nodecirc{$p_2$}}
  \begin{columns}
    \begin{column}{0.45\textwidth}
      Step 3: construct transitions from~\nodecirc{$p_2$}: compute Brzozowski derivatives of 
      \[p_2=\tystring^{*}\cdot \tyeven \cdot(\tyint \cdot \tystring^{*} \cdot \tyeven)^{*}\,.\]

      \only<1-2>{\includegraphics[width=0.9\textwidth,trim={1.4cm 1.2cm 1.4cm 0.8cm},clip=true]{reclojure-2025-sink-example-pre-p2}}%
      \only<3>{\includegraphics[width=0.9\textwidth,trim={1.4cm 1.2cm 1.4cm 0.8cm},clip=true]{reclojure-2025-sink-example-p2}}
    \end{column}
    \begin{column}{0.55\textwidth}
      \begin{itemize}
      \item<1->{What are the \emph{first} types of $p_2$: $\{\tystring,\tyeven\}$.}
      \item<2->{Partition $\Sigma$ by $\{\tystring,\tyeven\} \to \Pi$.}
      \item<3->{For each $\typevar \in \Pi$ compute $\deriv{\typevar}{p_2}$
        \begin{itemize}
        \item $\deriv{\tystring\cap\tyeven}{p_2} = p_3$ (NEW)
        \item $\deriv{\tystring\cap\tynot{\tyeven}}{p_2} = p_2$
        \item $\deriv{\tynot{\tystring}\cap\tyeven}{p_2} = p_0$
        \item $\deriv{\tynot{\tystring}\cap\tynot{\tyeven}}{p_2} = p_1$
        \end{itemize}
      }
      \end{itemize}
    \end{column}
  \end{columns}
\end{frame}

\begin{frame}{RTE Construction  of \nodecirc{$p_2$}}
  \begin{columns}
    \begin{column}{0.5\textwidth}
      \includegraphics[width=0.9\textwidth,trim={1.4cm 1.2cm 1.4cm 0.8cm},clip=true]{reclojure-2025-sink-example-p2}

      \begin{align*}
        p_0 &= (int \cdot str^{*} \cdot even)^{*}\\
        p_1 &= \emptyset\\
        p_2 &= \tystring^{*} \cdot \tyeven \cdot (\tyint \cdot \tystring^{*} \cdot \tyeven)^{*}\\
        p_3 &= \tystring^{*}\!\!\cdot\! \tyeven\!\cdot\! (\tyint\! \cdot\! \tystring^{*}\!\!\cdot\! \tyeven)^{*}\\
        &\quad\quad\reor (\tyint\! \cdot\! \tystring^{*}\!\!\cdot\!\! \tyeven)^{*}
      \end{align*}

    \end{column}
    \begin{column}{0.5\textwidth}
      \begin{align*}
        t_1 &= \Sigma\\
        t_2 &= \tyint  \\
        t_3 &= \compl{\tyint}\\
        t_4 &= \tystring \cap \tyeven\\
        t_5 &= \compl{\tystring} \cap \tyeven\\
        t_6 &= \tystring \cap \compl{\tyeven}\\
        %%t_7 &= (\tyint \cap \compl{\tyeven}) \cup (\tystring \cap \compl{\tyeven})\\
        %%t_8 &=\compl{\tyint} \cap \compl{\tystring} \cap \tyeven\\
        %%t_9 &=(\tyint \cap \tyeven) \cup (\tystring \cap \tyeven)\\
        t_{10} &=\compl{str} \cap \compl{even}\\
        %%t_{11} &=\compl{int} \cap \compl{str} \cap \compl{even}
      \end{align*}
    \end{column}
  \end{columns}
\end{frame}




\begin{frame}{RTE Construction of \nodecirc{$p_3$}}
  \begin{columns}
    \begin{column}{0.45\textwidth}
      Step 4: construct transitions from~\nodecirc{$p_3$}: compute Brzozowski derivatives of 
      \begin{align*}
        p_3&=\tystring^{*}\!\!\cdot\! \tyeven\!\cdot\! (\tyint\! \cdot\! \tystring^{*}\!\!\cdot\! \tyeven)^{*}\\
        &\quad\quad\reor (\tyint\! \cdot\! \tystring^{*}\!\!\cdot\!\! \tyeven)^{*}
      \end{align*}
      \only<1-2>{\includegraphics[width=0.9\textwidth,trim={1.4cm 1.2cm 1.4cm 0.8cm},clip=true]{reclojure-2025-sink-pre-p3}}%
      \only<3->{\includegraphics[width=0.9\textwidth,trim={1.4cm 1.2cm 1.4cm 0.8cm},clip=true]{reclojure-2025-sink-example-p3}}
    \end{column}
    \begin{column}{0.55\textwidth}
      \begin{itemize}
      \item<1->{What are the \emph{first} types of $p_3$: $\{\tystring,\tyint,\tyeven\}$.}
      \item<2->{Partition $\Sigma$ by $\{\tystring,\tyint,\tyeven\} \to \Pi$.}
      \item<3->{For each $\typevar \in \Pi$ compute $\deriv{\typevar}{p_3}$
        \begin{itemize}
        \item $\deriv{\tystring\tyand\tyint\tyand\tyeven}{p_3} = \deriv{\emptyset}{p_3}$ \textcolor{unsatisfiable}{;unsatisfiable}
        \item $\deriv{\tystring\tyand\tyint\tyand\tynot{\tyeven}}{p_3} = \deriv{\emptyset}{p_3}$ \textcolor{unsatisfiable}{;unsatisfiable}
        \item $\deriv{\tystring\tyand\tynot{\tyint}\tyand\tyeven}{p_3} = p_3$
        \item $\deriv{\tystring\tyand\tynot{\tyint}\tyand\tynot{\tyeven}}{p_3} = p_2$
        \item $\deriv{\tynot{\tystring}\tyand\tyint\tyand\tyeven}{p_3} = p_3$
        \item $\deriv{\tynot{\tystring}\tyand\tyint\tyand\tynot{\tyeven}}{p_3} = p_2$
        \item $\deriv{\tynot{\tystring}\tyand\tynot{\tyint}\tyand\tyeven}{p_3} = p_0$
        \item $\deriv{\tynot{\tystring}\tyand\tynot{\tyint}\tyand\tynot{\tyeven}}{p_3} = p_1$

        \end{itemize}
      }
      \end{itemize}
    \end{column}
  \end{columns}
\end{frame}

\begin{frame}{RTE Construction}
  \begin{columns}
    \begin{column}{0.5\textwidth}
      \includegraphics[width=\textwidth,trim={1.4cm 1.2cm 1.4cm 0.8cm},clip=true]{reclojure-2025-sink-example-2}
    \end{column}
    \begin{column}{0.5\textwidth}
      \begin{align*}
        t_1 &= \Sigma\\
        t_2 &= int  \\
        t_3 &= \compl{int}\\
        t_4 &= str \cap even\\
        t_5 &= \compl{str} \cap even\\
        t_6 &= str \cap \compl{even}\\
        t_7 &= (int \cap \compl{even}) \cup (str \cap \compl{even})\\
        t_8 &=\compl{int} \cap \compl{str} \cap even\\
        t_9 &=(int \cap even) \cup (str \cap even)\\
        t_{10} &=\compl{str} \cap \compl{even}\\
        t_{11} &=\compl{int} \cap \compl{str} \cap \compl{even}
      \end{align*}
    \end{column}
  \end{columns}
\end{frame}
