\subsection{Determinism: Type Partitioning}

{  %% chapter slide
  \setbeamercolor{background canvas}{bg=sectioncolor}
\begin{frame}{\Challenge{4} Type Partitioning: Deterministic State Machines}

  How to assure DFAs are deterministic by construction?

  \begin{itemize}
  \item   Compute a \Emph{sufficient} partition of a given set of type designators,
  \item   ... assure that $\deriv{\typevar}{r}$ be computable,
  \item   ... even (especially) when subtype relation is unknown,
  \item   ... correct run-time checks despite curse of subtype ambiguity.
  \end{itemize}

\end{frame}
}



\begin{frame}{MDTD: Maximal Disjoint Type Decomposition}
  \begin{columns}
    \begin{column}{0.5\textwidth}
      \only<1,2>{\scalebox{0.98}{\input{pgf-a1-a8.tex}}}%
      \only<3->{\includegraphics[scale=0.18]{venn-x1-x13.png}}%
    \end{column}
    \begin{column}{0.5\textwidth}
      \only<2->{\scalebox{0.98}{\input{pgf-x1-x13.tex}}}%
    \end{column}
  \end{columns}
\end{frame}

\newsavebox\boxforestAnu
\begin{lrbox}{\boxforestAnu}
  \begin{minipage}{\textwidth}
\begin{forest}% braces \typevar
  [$\Sigma$
    [{$A_1$},
      [{$A_2$},
        [{$A_2  A_3$},
          [{$A_2 A_3 A_4$},
            [{$\underbrace{A_2 A_3 A_4}_{\typevar_1}$},edge=dashed ] ]
          [{$A_2 A_3 \compl{A_4}$}, edge=dashed
            [{$\underbrace{A_2 A_3 \compl{A_4}}_{\typevar_2}$}, edge=dashed ] ] ]
        [{$A_2  \compl{A_3}$}, edge=dashed
          [{$A_2 \compl{A_3} A_4 $}
            [{$\underbrace{A_2 \compl{A_3} A_4}_{\typevar_3}$} ] ]
          [{$A_2 \compl{A_3}~ \compl{A_4}$}, edge=dashed
            [{$\underbrace{A_2 \compl{A_3}~ \compl{A_4}}_{\typevar_4}$}, edge=dashed ] ] ] ]
      [{$A_1  \compl{A_2}$}, edge=dashed
        [{$\compl{A_2} A_3  $}
          [{$\compl{A_2} A_3 A_4 $}
            [{$\underbrace{\compl{A_2} A_3 A_4}_{\typevar_5} $} ] ]
          [{$\compl{A_2} A_3  \compl{A_4}$}, edge=dashed
            [{$\underbrace{\compl{A_2} A_3 \compl{A_4}}_{\typevar_6}$}, edge=dashed ] ] ]
        [{$A_1  \compl{A_2} ~ \compl{A_3}$}, edge=dashed
          [{$\compl{A_2} ~ \compl{A_3} A_4 $}
            [{$\underbrace{A_5}_{\typevar_7}$} ]
            [{$\underbrace{\compl{A_2}~ \compl{A_3} A_4 \compl{A_5}}_{\typevar_8}$}, edge=dashed ] ]
          [{$A_1 \compl{A_2}~ \compl{A_3}~ \compl{A_4}$}, edge=dashed
            [{$\underbrace{A_1 \compl{A_2} ~\compl{A_3}~ \compl{A_4}}_{\typevar_9}$}, edge=dashed  ] ] ] ] ]
    [{$\compl{A_1}$}, edge=dashed
      [{$\compl{A_1}$}, edge=dashed
        [{$\compl{A_1}$}, edge=dashed
          [{$\compl{A_1}$}, edge=dashed
            [{$\underbrace{\compl{A_1}}_{\typevar_{10}}$}, edge=dashed ] ] ] ] ] ]
\end{forest}
  \end{minipage}
\end{lrbox}


\newsavebox\boxforestA
\begin{lrbox}{\boxforestA}
  \begin{minipage}{\textwidth}
\begin{forest}% cut with A_5
  [$\Sigma$
    [{$A_1$},
      [{$A_2$},
        [{$A_2  A_3$},
          [{$A_2 A_3 A_4$},
            [{$A_2 A_3 A_4$},edge=dashed, edge label={node[midway,left,font=\scriptsize,color=red]{cut=$A_5$}} ] ]
          [{$A_2 A_3 \compl{A_4}$}, edge=dashed
            [{$A_2 A_3 \compl{A_4}$}, edge=dashed ] ] ]
        [{$A_2  \compl{A_3}$}, edge=dashed
          [{$A_2 \compl{A_3} A_4 $}
            [{$A_2 \compl{A_3} A_4$} ] ]
          [{$A_2 \compl{A_3}~ \compl{A_4}$}, edge=dashed
            [{$A_2 \compl{A_3}~ \compl{A_4}$}, edge=dashed ] ] ] ]
      [{$A_1  \compl{A_2}$}, edge=dashed
        [{$\compl{A_2} A_3  $}
          [{$\compl{A_2} A_3 A_4 $}
            [{$\compl{A_2} A_3 A_4 $} ] ]
          [{$\compl{A_2} A_3  \compl{A_4}$}, edge=dashed
            [{$\compl{A_2} A_3 \compl{A_4}$}, edge=dashed ] ] ]
        [{$A_1  \compl{A_2} ~ \compl{A_3}$}, edge=dashed
          [{$\compl{A_2} ~ \compl{A_3} A_4 $}
            [{$A_5$} ]
            [{$\compl{A_2}~ \compl{A_3} A_4 \compl{A_5}$}, edge=dashed ] ]
          [{$A_1 \compl{A_2}~ \compl{A_3}~ \compl{A_4}$}, edge=dashed
            [{$A_1 \compl{A_2} ~\compl{A_3}~ \compl{A_4}$}, edge=dashed  ] ] ] ] ]
    [{$\compl{A_1}$}, edge=dashed
      [{$\compl{A_1}$}, edge=dashed
        [{$\compl{A_1}$}, edge=dashed
          [{$\compl{A_1}$}, edge=dashed
            [{$\compl{A_1}$}, edge=dashed ] ] ] ] ] ]
\end{forest}
  \end{minipage}
\end{lrbox}

\newsavebox\boxforestB
\begin{lrbox}{\boxforestB}
  \begin{minipage}{\textwidth}
\begin{forest}%cut with A_4
  [$\Sigma$
    [{$A_1$}
      [{$A_2$}
        [{$A_2  A_3$}
          [{$A_2 A_3 A_4$},edge label={node[midway,left,font=\scriptsize,color=red]{cut=$A_4$}} ]
          [{$A_2 A_3 \compl{A_4}$}, edge=dashed  ] ]
        [{$A_2  \compl{A_3}$}, edge=dashed
          [{$A_2 \compl{A_3} A_4$} ]
          [{$A_2 \compl{A_3}~ \compl{A_4}$}, edge=dashed ] ] ]
      [{$A_1  \compl{A_2}$}, edge=dashed
        [{$\compl{A_2} A_3  $}
          [{$\compl{A_2} A_3 A_4 $} ]
          [{$\compl{A_2} A_3  \compl{A_4}$}, edge=dashed ] ]
        [{$A_1  \compl{A_2} ~ \compl{A_3}$}, edge=dashed
          [{$\compl{A_2} ~ \compl{A_3} A_4 $} ]
          [{$A_1 \compl{A_2}~ \compl{A_3}~ \compl{A_4}$}, edge=dashed ] ] ] ]
    [{$\compl{A_1}$}, edge=dashed
      [{$\compl{A_1}$}, edge=dashed
        [{$\compl{A_1}$}, edge=dashed
          [{$\compl{A_1}$}, edge=dashed ] ] ] ] ]
\end{forest}
  \end{minipage}
\end{lrbox}



\newsavebox\boxforestC
\begin{lrbox}{\boxforestC}
  \begin{minipage}{\textwidth}
\begin{forest}%cut with A_3
  [$\Sigma$
    [{$A_1$}
      [{$A_2$}
        [{$A_2  A_3$},edge label={node[midway,left,color=red,font=\scriptsize]{cut=$A_3$}}] 
        [{$A_2  \compl{A_3}$}, edge=dashed ] ]
      [{$A_1  \compl{A_2}$}, edge=dashed,
        [{$\compl{A_2} A_3  $} ]
        [{$A_1  \compl{A_2} ~ \compl{A_3}$}, edge=dashed ] ] ]
    [{$\compl{A_1}$}, edge=dashed
      [{$\compl{A_1}$}, edge=dashed
        [{$\compl{A_1}$}, edge=dashed ] ] ] ]
\end{forest}
  \end{minipage}
\end{lrbox}




\newsavebox\boxforestD
\begin{lrbox}{\boxforestD}
  \begin{minipage}{\textwidth}
\begin{forest}
  [$\Sigma$
    [{$A_1$}
      [{$A_2$},edge label={node[midway,left,color=red,font=\scriptsize]{cut=$A_2$}}]
      [{$A_1  \compl{A_2}$}, edge=dashed] ]
    [{$\compl{A_1}$}, edge=dashed
      [{$\compl{A_1}$}, edge=dashed ] ] ]
\end{forest}
  \end{minipage}
\end{lrbox}



\newsavebox\boxforestE
\begin{lrbox}{\boxforestE}
  \begin{minipage}{\textwidth}
\begin{forest}
  [$\Sigma$
    [{$A_1$},edge label={node[midway,left,color=red,font=\scriptsize]{cut=$A_1$}}]
    [{$\compl{A_1}$}, edge=dashed ] ]
\end{forest}
  \end{minipage}
\end{lrbox}



\begin{frame}[t]{$MDTD( \{A_1, A_2, \ldots, A_5\})\to\Pi$}
  \only<1>{\scalebox{2.0}{\usebox\boxforestE}
    
    \medskip

    Start with $\Sigma=Universe$, and intersect with $A_1$ and $\compl{A_1}$.
  }
  \only<2>{%
    \begin{columns}
      \begin{column}{0.6\textwidth}
        \scalebox{1.5}{\usebox\boxforestD}%    
      \end{column}
      \begin{column}{0.4\textwidth}
        {\scalebox{0.7}{\input{pgf-a1-a8.tex}}}%
      \end{column}
    \end{columns}%
      
  Intersect each leaf with $A_2$ and $\compl{A_2}$.
  \begin{align*}
    A_2 \subset A_1 &\implies A_1\cap A_2 = A_2 & SIMPLIFY\\
    A_2 \subset A_1 &\implies \compl{A_1}\cap A_2 = \emptyset & PRUNE\\
    A_2 \subset A_1 &\implies \compl{A_1}\cap \compl{A_2} = \compl{A_1} & SIMPLIFY
  \end{align*}
  }
  \only<3>{%
    \begin{columns}
      \begin{column}{0.6\textwidth}
        \scalebox{1.1}{\usebox\boxforestC}
      \end{column}
      \begin{column}{0.4\textwidth}
        {\scalebox{0.7}{\input{pgf-a1-a8.tex}}}%
      \end{column}
    \end{columns}%
      
    
    \medskip
    
    Intersect each leaf with $A_3$ and $\compl{A_3}$.
    
    \begin{align*}
      A_3 \subset A_1 &\implies \compl{A_1} A_3 = \emptyset & PRUNE
    \end{align*}
  }
  \only<4>{\scalebox{0.8}{\usebox\boxforestB}
    
    \medskip
    
    Intersect each leaf with $A_4$ and $\compl{A_4}$.
  }
  \only<5>{\scalebox{0.7}{\usebox\boxforestA}
    
    \medskip
    
    Intersect each leaf with $A_5$ and $\compl{A_5}$.
    
    \begin{align*}
      A_2 \cap A_5 = \emptyset &\implies A_2A_3A_4A_5 = \emptyset & PRUNE\\
      A_5 \subset A_4 &\implies A_1\compl{A_2}~\compl{A_3}~\compl{A_2} \cap A_5=\emptyset & PRUNE\\
    \end{align*}
  }
  \only<6>{\scalebox{0.7}{\usebox\boxforestAnu}

    \medskip

    $MDTD=\Pi \xrightarrow{\leq O(2^n)} \{\typevar_1, \typevar_2, \ldots, \typevar_{10}\} = \{A_2 A_3 A_4,~ A_2 A_3 \compl{A_4}, ~\ldots,~ \compl{A_1}\}$.
  }%
  \only<7>{\scalebox{0.7}{\usebox\boxforestAnu}
    
    \medskip
    
    \begin{itemize}
    \item \Emph{By Construction:} either $\typevar_i \subseteq A_j$ or $\typevar_i \subseteq \tynot{A_j}$.
    \item \Emph{By Construction:} $\{\typevar_1, \typevar_2, \ldots, \typevar_{10}\}$ mutually disjoint.
    \item \Emph{By Construction:} $\bigcup_{i=1}^5\typevar_i=\Sigma$.
    \item Exactly the properties we need to compute $\deriv{\typevar}{\singleton{A_i}}$.
    \end{itemize}
  }
\end{frame}


\begin{frame}{Non-determinism by subtype}
  \begin{columns}[T]
    \begin{column}{0.4\textwidth}
      \centering
      
      \begin{align*}
        Int&\subseteq Number\\
        Int &\cap Number \neq \emptyset
      \end{align*}%
      \scalebox{0.8}{\input{tikz-int-float}}%
    \end{column}%
    \begin{column}{0.6\textwidth}
      \only<1>{\scalebox{0.9}{\input{fig-nfa-subtype-3}}}%
      \only<2>{\scalebox{0.8}{\input{fig-nfa-subtype-disjoint}}}%
    \end{column}
  \end{columns}
\end{frame}


\begin{frame}{Non-determinism by \code{SSatisfies}}
  \begin{columns}[T]
    \begin{column}{0.35\textwidth}
      \centering
      \begin{align*}
        Odd&\subseteq Int &\text{unknown}\\
        Odd&~\cap~ !Int = \emptyset &\text{unknown}
      \end{align*}
      \scalebox{1.0}{\input{tikz-int-odd.tex}}%
    \end{column}%
    \begin{column}{0.7\textwidth}
      \only<1>{\scalebox{0.9}{\input{fig-nfa-satisfies}}}%
      \only<2>{\scalebox{0.8}{\input{fig-nfa-satisfies-intersections}}}%
    \end{column}
  \end{columns}
\end{frame}




% Double and Odd + Double and Even


% classes A and B
% if A and B are final, then they are disjoint
% if A and B are abstract, there MIGHT be a common subclass
% if A and B are in a subtype relation, then either A &! B or B & !A is empty


\newsavebox\classbox
\begin{lrbox}{\classbox}
  \begin{minipage}{5cm}
    %% dont re-indent this file
\begin{lstlisting}[style=reclojureClojure]
(isa? Integer Number) ;; true
(isa? Number Integer) ;; false

(:flags (refl/type-reflect Number))
;; --> #{:public :abstract}

(:flags (refl/type-reflect Integer))
;; --> #{:public :final}

(:flags (refl/type-reflect IPersistentVector)) 
;; --> #{:interface :public :abstract}
\end{lstlisting}

  \end{minipage}
\end{lrbox}

\begin{frame}{Unsatisfiable Transitions}

  \scalebox{0.8}{\input{fig-4-unsat}}
  \begin{itemize}
  \item If we (compile-time) determine a type is empty, then the transition is \Emph{unsatisfiable}.
  \item Thus we \Emph{can eliminate} the transition and unreachable states.

  \end{itemize}
\end{frame}


\begin{frame}{Indeterminate Transitions}

  \begin{tabular}{cc}
    \scalebox{0.7}{\input{fig-4-str-odd}}&\scalebox{0.7}{\input{fig-4-int-odd}}\\
    Indeterminate, \textcolor{red}{Unsatisfiable} & Indeterminate, \textcolor{greeny}{Satisfiable}
  \end{tabular}
  \only<1>{%
  \begin{itemize}
  \item   If we cannot determine a type is empty, the transition may
    \Emph{still be unsatisfiable}. 
  \item  However, we \Emph{cannot eliminate}    the transition and unreachable states.
  \end{itemize}
  }%
  \only<2>{%
  \begin{itemize}
    \item We \Emph{can always} (run-time) determine \Emph{type membership}.
    \item DFAs with indeterminate transitions \Emph{correctly}
      match sequences in $O(n)$.
  \end{itemize}
  }
\end{frame}



