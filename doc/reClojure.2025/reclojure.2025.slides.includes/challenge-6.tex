%\subsection{Vacuity: Habitation Checks}

{  %% chapter slide
  \setbeamercolor{background canvas}{bg=sectioncolor}
\begin{frame}{\Challenge{6} How to detect habitation/vacuity in a DFA}

  \centering
  \includegraphics[width=\linewidth]{p4-p3}
\end{frame}
}

\newsavebox\demoAbox
\begin{lrbox}{\demoAbox}
  \begin{minipage}{8cm}
    %% dont re-indent this file
\begin{lstlisting}[style=reclojureScala]
def large(n:Any):Boolean = {
  n match {
    case n: Int => n > 127
    case _ => false
  }
}

def even(n:Any):Boolean = {
  n match {
    case n: Int => n % 2 == 0
    case _ => false
  }
}
\end{lstlisting}

  \end{minipage}
\end{lrbox}

\newsavebox\demoBbox
\begin{lrbox}{\demoBbox}
  \begin{minipage}{8cm}
    %% dont re-indent this file
\begin{lstlisting}[style=reclojureClojure]
(def pat-1
  '(:* (:cat String
             (:* (? even)))))
(def pat-2
  '(:* (:cat String
             (:* (and Integer 
                      (? even))))))
(def pat-3
  '(:* (:cat String
             (:* (and Integer
                      (? even)
                      (? large))))))
\end{lstlisting}

  \end{minipage}
\end{lrbox}

\newsavebox\demoCbox
\begin{lrbox}{\demoCbox}
  \begin{minipage}{8cm}
    %% dont re-indent this file
\begin{lstlisting}[style=reclojureScala]
val data:Seq[Any] = 
  Seq("Lorem",
      "ipsum", 2, 4, 8,
      "dolor", 10, 20)

pat1.contains(data)
// --> true

pat2.contains(data)
// --> true

pat3.contains(data)
// --> false
\end{lstlisting}

  \end{minipage}
\end{lrbox}



\begin{frame}{Setup}{Declaration Scala Code}
  \begin{columns}
    \begin{column}{0.5\textwidth}
      \usebox\demoBbox
    \end{column}
    \begin{column}{0.5\textwidth}
      \only<1>{\usebox\demoAbox}%
      \only<2>{\usebox\demoCbox}%
    \end{column}
  \end{columns}
\end{frame}

\newsavebox\demoDbox
\begin{lrbox}{\demoDbox}
  \begin{minipage}{8cm}
    %% dont re-indent this file
\begin{lstlisting}[style=reclojureScala]
pat1 = (S ++ Even.*).*
dfaView(pat1.toDfa())
\end{lstlisting}

  \end{minipage}
\end{lrbox}
\newsavebox\demoEbox
\begin{lrbox}{\demoEbox}
  \begin{minipage}{8cm}
    %% dont re-indent this file
\begin{lstlisting}[style=reclojureClojure]
(let [pat-2 '(:* (:cat String (:* (and Integer (? even)))))]
  (dfa-to-dot pat-2 :view true))
\end{lstlisting}

  \end{minipage}
\end{lrbox}
\newsavebox\demoFbox
\begin{lrbox}{\demoFbox}
  \begin{minipage}{8cm}
    %% dont re-indent this file
\begin{lstlisting}[style=reclojureScala]
pat3 = (S ++ (I & Even & Large).*).*
dfaView(pat3.toDfa())
\end{lstlisting}

  \end{minipage}
\end{lrbox}
\newsavebox\demoGbox
\begin{lrbox}{\demoGbox}
  \begin{minipage}{8cm}
    %% dont re-indent this file
\begin{lstlisting}[style=reclojureScala]
val pat1 = (S ++ Even.*).*
val pat2 = (S ++ (I & Even).*).*
val pat3 = (S ++ (I & Even & Large).*).*
\end{lstlisting}

  \end{minipage}
\end{lrbox}

\newsavebox\demoHbox
\begin{lrbox}{\demoHbox}
  \begin{minipage}{8cm}
    %% dont re-indent this file
\begin{lstlisting}[style=reclojureClojure]
(let [pat-1 '(:* (:cat String (? even)))
      pat-2 '(:* (:cat String (and Integer
                                   (? even))))
      diff-21 (gns/- pat-2 pat-1)]

 ;; emptyset if pat2 subset of pat1
 (dfa-to-dot diff-21 :view true))
\end{lstlisting}

  \end{minipage}
\end{lrbox}


\newsavebox\demoIbox
\begin{lrbox}{\demoIbox}
  \begin{minipage}{8cm}
    %% dont re-indent this file
\begin{lstlisting}[style=reclojureScala]
val pat1 = (S ++ Even.*).*
val pat2 = (S ++ (I & Even).*).*
val pat3 = (S ++ (I & Even & Large).*).*

// emptyset if pat1 subset of pat2
val diff12 = pat1 - pat2
dfaView(diff12.toDfa())
\end{lstlisting}

  \end{minipage}
\end{lrbox}

\newsavebox\demoJbox
\begin{lrbox}{\demoJbox}
  \begin{minipage}{8cm}
    %% dont re-indent this file
\begin{lstlisting}[style=reclojureClojure]
(let [pat-1 '(:* String (:* (? even)))
      pat-2 '(:* (:cat String (and Integer (? even))))
      pat-3 '(:* (:cat String (and Integer
                                   (? even) 
                                   (? large))))
      diff-32 (template (and ~pat-3 (not ~pat-2)))]

  ;; emptyset if pat3 subset of pat2
  (dfa-to-dot diff-32 :view true))
\end{lstlisting}

  \end{minipage}
\end{lrbox}

\newsavebox\demoKbox
\begin{lrbox}{\demoKbox}
  \begin{minipage}{8cm}
    %% dont re-indent this file
\begin{lstlisting}[style=reclojureClojure]
(let [pat-1 '(:* (:cat String (:* (? even))))
      pat-2 '(:* (:cat String (:* (and Long (? even)))))
      pat-3 '(:* (:cat String (:* (and Long (? even) (? large)))))
      diff-23 (gns/- pat-2 pat-3)]

   ;; Example of element in pat2 but not pat3
   (dfa-to-dot diff-23 :view true))
\end{lstlisting}

  \end{minipage}
\end{lrbox}




\begin{frame}{DFAs}
  \only<1>{\usebox\demoDbox}%
  \only<2>{\usebox\demoEbox}%
  \only<3>{\usebox\demoFbox}%

  \only<1>{\Large
    \begin{align*}
      t_0 &= \text{\code{S}}\\
      t_1 &= \text{\code{S | Even}}
    \end{align*}}%
  \only<2>{\Large
    \begin{align*}
      t_0 &= \text{\code{S}}\\
      t_1 &= \text{\code{S | (I \& Even)}}
    \end{align*}}%
  \only<3>{\Large
    \begin{align*}
      t_0 &= \text{\code{S}}\\
      t_1 &= \text{\code{S | (I \& Even \& Large)}}
    \end{align*}}

  \only<1>{\centering\includegraphics[width=8cm]{pat1}}%
  \only<2>{\centering\includegraphics[width=8cm]{pat2}}%
  \only<3>{\centering\includegraphics[width=8cm]{pat3}}%
\end{frame}

\begin{frame}{Inclusion Checks}
  \usebox\demoGbox

  By inspection we see that \[\sem{pat3} \subsetneq \sem{pat2} = \sem{pat1}\,.\]
  Can we determine this programmatically?
\end{frame}



\begin{frame}{Inclusion Checks}


  \only<1>{\[\sem{pat3} \subsetneq \underbrace{\sem{pat2} = \sem{pat1}}_{\sem{pat2}~ \subset~ \sem{pat1}~ =~ YES}\]}%
  \only<2>{\[\sem{pat3} \subsetneq \underbrace{\sem{pat2} = \sem{pat1}}_{\sem{pat2}~ \supset~ \sem{pat1}~ =~ DONT-KNOW}\]}%
  \only<3>{\[\underbrace{\sem{pat3} \subsetneq \sem{pat2}}_{YES} = \sem{pat1}\]}%
  \only<4>{\[\sem{pat2} \cap \overline{\sem{pat3}}\]}%

  \begin{columns}
    \begin{column}{0.5\textwidth}
      \only<1>{\usebox\demoHbox}%
      \only<2>{\usebox\demoIbox}%
      \only<3>{\usebox\demoJbox}%
      \only<4>{\usebox\demoKbox}%
    \end{column}
    \begin{column}{0.5\textwidth}
      \only<1>{
        \begin{align*}
          t_0 &= \text{\code{S}}\\
          t_1 &= \text{\code{S | (I \& Even)}}
        \end{align*}
        %                          [trim={left bottom right top},clip]
        \includegraphics[width=8cm,trim={0 0 0 15mm},clip]{p2-p1}
      }%
      \only<2>{
    \begin{align*}
      t_0 &= \text{\code{S}}\\
      t_1 &= \text{\code{!I \& !S \& Even}} && Indeterminate\\
      t_2 &= \text{\code{S | (I \& Even)}}\\
      t_3 &= \text{\code{S | Even}}
    \end{align*}
    %                          [trim={left bottom right top},clip]
    \includegraphics[width=8cm,trim={0 0 0 15mm},clip]{p1-p2}
      }%
      \only<3>{
    \begin{align*}
      t_0 &= \text{\code{S}}\\
      t_1 &= \text{\code{S | (I \& Even, Large)}}
    \end{align*}
    %                          [trim={left bottom right top},clip]
    \includegraphics[width=8cm,trim={0 0 0 15mm},clip]{p3-p2}
      }%
      \only<4>{
        \begin{align*}
          t_0 &= \text{\code{S}}\\
          t_1 &= \text{\code{I \& Even \& !Large}}\\
          t_2 &= \text{\code{S | (I \& Even \& Large)}}\\
          t_3 &= \text{\code{S | (I \& Even)}}\\
        \end{align*}
        %                          [trim={left bottom right top},clip]
        \includegraphics[width=8cm,trim={0 0 0 15mm},clip]{p2-p3}
      }
    \end{column}
  \end{columns}
\end{frame}


\begin{frame}{Habitation Check}

  Is the language of the DFA inhabited?
  \begin{itemize}
  \item YES -- satisfiable path to accepting state
  \item NO -- no path to accepting state
  \item DONT-KNOW -- only indeterminate paths to accepting state
  \end{itemize}

  \includegraphics[width=\linewidth]{p4-p3}

\end{frame}


\begin{frame}{Habitation Check}

  How do Dijkstra weights work?

  \only<1>{\scalebox{0.8}{\documentclass{standalone}
  \usepackage{tikz}
  \usetikzlibrary{arrows.meta, automata, bending, positioning, shapes.misc}
  \tikzstyle{automaton}=[shorten >=1pt, >={Stealth[bend,round]}, initial text=]
  \tikzstyle{accepting}=[double]

\begin{document}
\begin{tikzpicture}[automaton, auto, thick]
  \node[state,initial,rounded rectangle] (0) {$0$};
  \node[state,color=red,rounded rectangle] (1) [above right=7mm and 50mm of 0] {$1$};
  \node[state,text=black,accepting,rounded rectangle] (2) [right=50mm of 1] {$2$};
  \node[state,text=black,rounded rectangle] (3) [below right=7mm and 35mm of 0] {$3$};
  \node[state,text=black,rounded rectangle] (4) [right=35mm of 3] {$4$};
  \path[->] (0) edge[color=indeterminate, dotted, line width=3pt] node {$Int \cap Odd$} (1);
  \path[->] (0) edge[swap] node {$\overline{Int}$} (3);
  \path[->] (1) edge[color=indeterminate, dotted, line width=3pt]  node[pos=.5] {$Int$} (2);
  \path[->] (3) edge    node {$String$} (4);
  \path[->] (4) edge[swap]    node {$Float$} (2);
\end{tikzpicture}
\end{document}
}}%
  \only<2,3>{\scalebox{0.8}{\documentclass{standalone}
  \usepackage{tikz}
  \usetikzlibrary{arrows.meta, automata, bending, positioning, shapes.misc}
  \tikzstyle{automaton}=[shorten >=1pt, >={Stealth[bend,round]}, initial text=]
  \tikzstyle{accepting}=[double]

\begin{document}
\begin{tikzpicture}[automaton, auto, thick]
  \node[state,initial,rounded rectangle] (0) {$0$};
  \node[state,color=red,rounded rectangle] (1) [above right=7mm and 50mm of 0] {$1$};
  \node[state,text=black,accepting,rounded rectangle] (2) [right=50mm of 1] {$2$};
  \node[state,text=black,rounded rectangle] (3) [below right=7mm and 35mm of 0] {$3$};
  \node[state,text=black,rounded rectangle] (4) [right=35mm of 3] {$4$};
  \path[->] (0) edge[color=indeterminate, dotted, line width=3pt] node {$Int \cap Odd$} (1);
  \path[->] (0) edge[swap] node {$\overline{Int} ~ \textcolor{blue}{[W\!\!=\!1]}$} (3);
  \path[->] (1) edge[color=indeterminate, dotted, line width=3pt]  node[pos=.5] {$Int$ \textcolor{blue}{$[W\!\!=\!1]$}} (2);
  \path[->] (3) edge    node {$String ~ \textcolor{blue}{[W\!\!=\!1]}$} (4);
  \path[->] (4) edge[swap]    node {$Float ~ \textcolor{blue}{[W\!\!=\!1]}$} (2);
\end{tikzpicture}
\end{document}
}}%
  \only<4->{\scalebox{0.8}{\documentclass{standalone}
  \usepackage{tikz}
  \usetikzlibrary{arrows.meta, automata, bending, positioning, shapes.misc}
  \tikzstyle{automaton}=[shorten >=1pt, >={Stealth[bend,round]}, initial text=]
  \tikzstyle{accepting}=[double]

\begin{document}
\begin{tikzpicture}[automaton, auto, thick]
  \node[state,initial,rounded rectangle] (0) {$0$};
  \node[state,color=red,rounded rectangle] (1) [above right=7mm and 50mm of 0] {$1$};
  \node[state,text=black,accepting,rounded rectangle] (2) [right=50mm of 1] {$2$};
  \node[state,text=black,rounded rectangle] (3) [below right=7mm and 35mm of 0] {$3$};
  \node[state,text=black,rounded rectangle] (4) [right=35mm of 3] {$4$};
  \path[->] (0) edge[color=indeterminate, dotted, line width=3pt] node {$Int \cap Odd  ~ \textcolor{red}{[W\!\!=\!5]}$} (1);
  \path[->] (0) edge[swap] node {$\overline{Int} ~ \textcolor{blue}{[W\!\!=\!1]}$} (3);
  \path[->] (1) edge[color=indeterminate, dotted, line width=3pt]  node[pos=.5] {$Int$ \textcolor{blue}{$[W\!\!=\!1]$}} (2);
  \path[->] (3) edge    node {$String ~ \textcolor{blue}{[W\!\!=\!1]}$} (4);
  \path[->] (4) edge[swap]    node {$Float ~ \textcolor{blue}{[W\!\!=\!1]}$} (2);
\end{tikzpicture}
\end{document}
}}%

  \begin{itemize}
  \item<2->{Satisfiable transitions have weight \textcolor{blue}{$W\!=1$}.}
  \item<3->{Unsatisfiable transitions have weight \textcolor{red}{$W\!=\infty$}.}
  \item<4->{Indeterminate transitions have weight \textcolor{red}{$W\!=5$} (number of states).}
  \item<5>{Shortest path is $\nodecirc{0} \xrightarrow{1} \nodecirc{3} \xrightarrow{1} \nodecirc{4}  \xrightarrow{1} \nodecirc{2}$; length is $3$\\
  Longer path is $\nodecirc{0}\xrightarrow{5}\nodecirc{1}\xrightarrow{1}\nodecirc{2}$; length is $6$}
  \end{itemize}
\end{frame}



