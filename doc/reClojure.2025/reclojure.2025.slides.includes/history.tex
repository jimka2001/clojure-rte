\section{History}
{  %% chapter slide
  \setbeamercolor{background canvas}{bg=chaptercolor}
\begin{frame}{History}
  \centering
  %% image from https://patimes.org/history-and-its-relevance-to-public-administration/  
  \includegraphics[width=0.8\textwidth]{reclojure-history.jpg}
\end{frame}
}
\begin{frame}{Publication History}
  \begin{itemize}
  \item Regular Type Expressions (RTEs) ELS 2016
  \item PhD Thesis 2018 for theoretical, implementation, and performance details
    \url{https://www.lrde.epita.fr/wiki/Publications/newton.18.phd}
    
  \item \textit{A Portable, Simple, Embeddable Type System} [Newton, Pommellet] 2021 ELS

  \item \textit{An Elegant and Fast Algorithm for Partitioning Types} [Newton] 2023 ELS
  \item \textit{Type-Checking Heterogeneous Sequences in a Simple Embeddable Type System} [Newton] 2025 PADL
  \item   RTE Available:

    \medskip
    
  \begin{tabular}{ll}
    Scala & \url{https://github.com/jimka2001/scala-rte}\\
    Clojure & \url{https://github.com/jimka2001/clojure-rte}\\
    Python & \url{https://github.com/jimka2001/python-rte}\\
    Common Lisp & \url{https://github.com/jimka2001/cl-rte}    
  \end{tabular}
  \end{itemize}
\end{frame}


\begin{frame}{Goal}
  \begin{itemize}
  \item Efficiently recognize \emph{regular patterns} in mixed-type sequences.
  \item Supported in various programming languages:
    \begin{itemize}
    \item Scala as \code{Seq[Any]}
    \item Python as \code{list}, \code{generator}
    \item Clojure as \code{seq}
    \item Common Lisp as \code{SEQUENCE}
    \end{itemize}

  \end{itemize}
\end{frame}
